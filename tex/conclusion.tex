\section{Conclusion}

\begin{frame}{Summary}

\end{frame}

\begin{frame}{Challenges (\& Solutions?)}

Unfortunately, autoencoder design and training itself typically
requires skilled human input; AutoMap cannot entirely sidestep the
need for manual labor. Indeed, the question of how to adapt autoen-
coder architecture and training to a less well-understood domain is
nontrivial. For example, questions such as “What should the size
of the bottleneck be for a bottlenecked autoencoder?” or “What
type of noise should a denoising autoencoder be trained with?”
will need to be addressed in any application of AutoMap. It should
also be noted that the success of AutoMap in any problem domain
depends on the computational capacity to generate large amounts
of training data through direct evolution and a willingness to accept
the computational cost of performing a forward pass through the
autoencoder component for each fitness evaluation when evolving
with a learned genotype-phenotype map. Domains where evolu-
tion with pre-existing encodings generate poor solutions, repeated
cycles of autoencoder training and generation of new training data
via evolution might be necessary to yield satisfactory performance.

chicken \& egg: iterative bootstrapping

\end{frame}

\begin{frame}{Next Steps}

demonstrate on more challenging problem

soft-bodied robots?

\end{frame}

\begin{frame}{Next Steps}

put in conversation with ML-evolvability synthesis

citations

call me, beep me, you know how to reach me \dots

or just have some cool ideas and cite me

\end{frame}
